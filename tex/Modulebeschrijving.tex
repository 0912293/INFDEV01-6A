\section*{Module description}
\begin{tabularx}{\textwidth}{|>{\columncolor{lichtGrijs}} p{.26\textwidth}|X|}
	\hline
	\textbf{Module name:} & \modulenaam\\
	\hline
	\textbf{Module code: }& \modulecode\\
	\hline
	\textbf{Study points \newline and hours of effort for full-time students:} & This module gives \stdPunten, in correspondance with 112 hours:
	\begin{itemize}
		\item 2 x 8 hours frontal lecture
		\item 3 x 8 hours self-study for the theory
		\item the rest is self-study for the practicum
	\end{itemize} \\
	\hline
	\textbf{Examination:} & Written examination and practical assignment (with oral check) \\
	\hline
	\textbf{Course structure:} & Lectures \\
	\hline
	\textbf{Prerequisite knowledge:} & Object oriented programming \\
	\hline
	\textbf{Learning tools:} & \begin{itemize}
			\item Book: \textit{Algorithms} (4rd edition); authors R. Sedgewick, K. Wayne
			\item Lesson slides (pdf): found on N@tschool
			\item Assignments, to be done at home (pdf): found on N@tschool
		\end{itemize} \\
	\hline
	\textbf{Connected to \newline competences:} & \begin{itemize}
			\item Realisation
		\end{itemize} \\
	\hline
	\textbf{Learning objectives:} &
		At the end of the course, the student:
			\begin{itemize}
				\item is familiar with concepts of data structures and algorithms [\texttt{KNOW}]
				\item can independently implement fundamental data structures and algorithms [\texttt{IMPL}]
				\item can analyze the efficiency of algorithms (in terms of time) [\texttt{AN}]
				\item can recognize the most effective approach (in terms of algorithms and data structures) to practical problems [\texttt{APP}]
			\end{itemize} \\
		
	\hline
\end{tabularx}
\newpage

\begin{tabularx}{\textwidth}{|>{\columncolor{lichtGrijs}} p{.26\textwidth}|X|}
	\hline
	\textbf{Content:}& \begin{itemize}[noitemsep]
		\item Performance analysis
		\item Basic data structures
		\item Sorting algorithms 
		\item Advanced data structures (trees, graphs)
		\item Path algorithms
		\item Dynamic programming
	\end{itemize} \\
	\hline
	\textbf{Course owners:} & \author\\
	\hline
	\textbf{Date:} & \today \\
	\hline
\end{tabularx}
\newpage
