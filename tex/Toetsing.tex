\section{Assessment}
	The course is tested with two exams: a practical assignment (accompanied by a brief oral check), and a written exam. The final grade is determined by the practical assignments. However, to receive the grade in the practical assignment you \textbf{must} have a sufficient (i.e. $\geq$ 5.5) grade in the written exam.

	\subsection{Theoretical examination}
		The theoretical exam will cover topics seen in class. The questions will be both theoretical and about code analysis, such as understanding what a code snippet does, determining its complexity, or finding mistakes in it.
		The exam lasts two lesson hours (100 minutes). No help is allowed during the exam.

	\subsection{Practical examination}
	The practical assignment must be done individually. \\
	You must upload your projects on Github and (only at the end) on N@tschool. \\
	The teachers must be added to the Github repository. \\
	The intermediate deadlines will be checked through the commits in Github.\\
	There will be oral checks to verify the authorship of code. \\
	The framework for the assignment comes only for .NET languages: allowed languages are C\# and F\#.  \\ 
	A detailed description of the practical assignment will be uploaded on N@tschool.

	\subsection{Herkansing}
	If one part of the assessment is not sufficient (theoretical and/or practical examination), then you can repeat that part in the following block:
	\begin{itemize}
	\item In week 10 of the following block you can repeat the written exam.
	\item The deadline for the delivery of the practical examination is at the end of week 9 of the following block.
	\end{itemize}
	
